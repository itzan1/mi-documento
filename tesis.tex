
\documentclass[twoside,11pt]{Latex/Classes/thesisUMSNH}
%         PUEDEN INCLUIR EN ESTE ESPACIO LOS PAQUETES EXTRA, O BIEN, EN EL ARCHIVO "PhDthesisPSnPDF.cls" EN "./Latex/Classes/"
\usepackage{blindtext}                        % Para insertar texto dummy, de ejemplo, pues.
\usepackage[round, sort, numbers]{natbib}  % Personalizar la bibliografía a gusto de cada quien
% Note:
% The \blindtext or \Blindtext commands throughout this template generate dummy text
% to fill the template out. These commands should all be removed when 
% writing thesis content.
\include{Latex/Macros/MacroFile1}           % Archivo con funciones útiles





%%%%%%%%%%%%%%%%%%%%%%%%%%%%%%%%%%%%%%%%%%%%%%%%%%%%%%%%%%%%%%%%%%%%%%%%%%%%%%%%
%                                   DATOS                                      %
%%%%%%%%%%%%%%%%%%%%%%%%%%%%%%%%%%%%%%%%%%%%%%%%%%%%%%%%%%%%%%%%%%%%%%%%%%%%%%%%
\title{ Fuente digital de alto voltaje, basado en embobinado de ignición y sistema Cockroft-Walton}
\author{Juan Francisco Verdugo Arredondo} 
\facultad{Facultad de Ciencias Físico Matemáticas}   
\escudofacultad{Latex/Classes/Escudos/fmed_grande} % Aquí ponen la ruta y nombre del escudo de su facultad, actualmente, la carpeta Latex/Classes/Escudos cuenta con los siguientes escudos:
% "fi_azul" Facultad de ingenieria en color azul
% "fi_negro" Facultad de ingenieria en color negro
% "fc_azul" Facultad de ciencias en color azul
% "fc_negro" Facultad de ciencias en color negro

\degree{Licenciado en Electrónica}       % Carrera
\director{Carlos Duarte Galván \\ Christian Valerio Lizárraga}                  % Director de tesis
%\tutor{Nombre  Tutor }                    % Tutor de tesis, si aplica
\degreedate{2017}                                     % Año de la fecha del examen
\lugar{Culiacán, Sinaloa}                        % Lugar

%\portadafalse                              % Portada en NEGRO, descomentar y comentar la línea siguiente si se quiere utilizar
\portadatrue                                % Portada en COLOR



%% Opciones del posgrado (descomentar si las necesitan)
	%\posgradotrue                                                    
	%\programa{programa de maestría y doctorado en ingeniería}
	%\campo{Ingeniería Eléctrica - Control}
	%% En caso de que haya comité tutor
	%\comitetrue
	%\ctutoruno{Dr. Emmet L. Brown}
	%\ctutordos{Dr. El Doctor}
%% Datos del jurado                             
	%\presidente{Dr. 1}
	%\secretario{Dr. 2}
	%\vocal{Dr. 3}
	%\supuno{Dr. 4}
	%\supdos{Dr. 5}
	%\institucion{el Instituto de Ingeniería, UNAM}

\keywords{tesis,Juan Francisco Verdugo Arredondo,Carlos Duarte Galvan,acelerador de partículas}            % Palablas clave para los metadatos del PDF
\subject{acelerador de partículas,Física de altas energías}                     % Tema para metadatos del PDF  

%%%%%%%%%%%%%%%%%%%%%%%%%%%%%%%%%%%%%%%%%%%%%%%%%%%%%
%                   PORTADA                         %
%%%%%%%%%%%%%%%%%%%%%%%%%%%%%%%%%%%%%%%%%%%%%%%%%%%%%
\begin{document}

\maketitle									% Se redefinió este comando en el archivo de la clase para generar automáticamente la portada a partir de los datos

%%%%%%%%%%%%%%%%%%%%%%%%%%%%%%%%%%%%%%%%%%%%%%%%%%%%%
%                  PRÓLOGO                          %
%%%%%%%%%%%%%%%%%%%%%%%%%%%%%%%%%%%%%%%%%%%%%%%%%%%%%
\frontmatter
\begin{dedication}
A mis padres, Martha Aidé Arredondo Solís y Francisco Verdugo Fierro.
A mis hermanos Dulce Esmeralda Verdugo Arredondo y Néstor Javier Espino Arredondo.
Por ayudarme a crecer en cada aspecto de mi vida.\\
Yo.
\end{dedication}
       % Comentar línea si no se usa
%\chapter*{}
%\pagenumbering{Roman}

\begin{acknowledgements}

También quiero agradecer a mis grandes maestros que me enseñaron que la sabiduría se encuentra en la motivación y el trabajo duro, Carlos Duarte Galvan, Cristian Valerio, Dr. Millan, etc..\\
A mi Universidad y sus instituciones, por abrirme las puertas a esta gran casa de estudio, al parque de innovación tecnológico, por haberme formado de la mejor manera posible y por la infinidad de oportunidades que se abren tras el conocimiento que se me ha otorgado. Sin mas ¡Gracias!
\end{acknowledgements}




   % Comentar línea si no se usa 
\include{Declaracion/Declaracion}           % Comentar línea si no se usa
%0
% Thesis Abstract -----------------------------------------------------


%\begin{abstractslong}    %uncommenting this line, gives a different abstract heading
\begin{abstracts}        %this creates the heading for the abstract page
El trabajo de esta tesis consiste en el desarrollo de la etapa de potencia de un acelerador de partículas lineal, el cual se centra en el diseño y construcción de una fuente de alto voltaje (HPS) basado en el sistema Cockcroft Walton  y un transformador de ignición capaz de suministrar 20 KV a 180 W, controlando el sistema digitalmente mediante un micro-controlador y comunicación Serial PC-HPS.\\☺

Durante el desarrollo de potencia se hizo el diseño, la simulación, la fabricación y la calibración de la electrónica y mecánica necesaria, optimizando así la precisión y costes del proyecto.\\*

Obteniendo como resultado un sistema de bajo coste, con el cual, mediante un computador se tiene la posibilidad de un control automatizado y  del sistema en general. 


\end{abstracts}
%\end{abstractlongs}


% ----------------------------------------------------------------------                   % Comentar línea si no se usa

%%%%%%%%%%%%%%%%%%%%%%%%%%%%%%%%%%%%%%%%%%%%%%%%%%%%%
%                   ÍNDICES                         %
%%%%%%%%%%%%%%%%%%%%%%%%%%%%%%%%%%%%%%%%%%%%%%%%%%%%%
%Esta sección genera el índice
\setcounter{secnumdepth}{3} % organisational level that receives a numbers
\setcounter{tocdepth}{3}    % print table of contents for level 3
\tableofcontents            % Genera el índice 
%: ----------------------- list of figures/tables ------------------------
\listoffigures              % Genera el ínidce de figuras, comentar línea si no se usa
\listoftables               % Genera índice de tablas, comentar línea si no se usa


%%%%%%%%%%%%%%%%%%%%%%%%%%%%%%%%%%%%%%%%%%%%%%%%%%%%%
%                   CONTENIDO                       %
%%%%%%%%%%%%%%%%%%%%%%%%%%%%%%%%%%%%%%%%%%%%%%%%%%%%%
% the main text starts here with the introduction, 1st chapter,...
\mainmatter
\def\baselinestretch{1.5}                   % Interlineado de 1.5

% this file is called up by thesis.tex
% content in this file will be fed into the main document
%----------------------- introduction file header -----------------------
%%%%%%%%%%%%%%%%%%%%%%%%%%%%%%%%%%%%%%%%%%%%%%%%%%%%%%%%%%%%%%%%%%%%%%%%%
%  Capítulo 1: Introducción- DEFINIR OBJETIVOS DE LA TESIS              %
%%%%%%%%%%%%%%%%%%%%%%%%%%%%%%%%%%%%%%%%%%%%%%%%%%%%%%%%%%%%%%%%%%%%%%%%%

%\chapter{Introducción}

%: ----------------------- HELP: latex document organisation
% the commands below help you to subdivide and organise your thesis
%    \chapter{}       = level 1, top level
%    \section{}       = level 2
%    \subsection{}    = level 3
%    \subsubsection{} = level 4
%%%%%%%%%%%%%%%%%%%%%%%%%%%%%%%%%%%%%%%%%%%%%%%%%%%%%%%%%%%%%%%%%%%%%%%%%
%                           Presentación                                %
%%%%%%%%%%%%%%%%%%%%%%%%%%%%%%%%%%%%%%%%%%%%%%%%%%%%%%%%%%%%%%%%%%%%%%%%%

\chapter{Resumen} % section headings are printed smaller than chapter names

El trabajo de esta tesis consiste en el desarrollo de una fuente de alto voltaje de bajo ruido para su uso en aplicaciones de un acelerador de partículas lineal.  Esta fuente está basada en la topología de los sistemas Cockcroft Walton \cite{Cockcroft}, ademas incluyendo  un sistema de inversor de voltaje y un rectificador del tipo multiplicador, así como también la introducción a un control del sistema con lazo cerrado e interfaces gráficas para el usuario donde mediante un computador se tiene la posibilidad de un control automatizado y  del sistema en general. 

Durante el desarrollo del sistema de alto voltaje se realizo el diseño del instrumento realizando una serie de simulaciones para encontrar el punto de operación optimo previo a la fabricación. Como resultado nuestro sisttema construido siguio fielmente los parametros de las simulaciones y puede generar alto voltaje a 10 W de potencia con un facror de ruido por debajo del 1$\%$, ademas se realizo una comparación con fuentes de alto voltaje comerciales implementadas en aceleradores de partículas y reactores nucleares de baja potencia.


\section{Antecedentes y justificación}


Uno de los mayores avances tecnológicos de la humanidad ha sido el desarrollo de aceleradores de partículas, ya que tienen aplicaciones en las áreas médicas, militares y alimentarias \cite{ProyectoLNLS5},\cite{neutrons1},\cite{leonard}. Los cuales aceleran los protones y electrones típicamente a energías de 10 MeV. Estos aceleradores son diseñados para funcionar de manera confiable produciendo haces de alta intensidad con un mínimo de intervención humana, he allí la meta en esta tesis.\\

En el pasado la radioterapia utilizaba agujas de radio o rayos gamma de cobalto radioactivo\cite{cobalt}, la desventaja de este tipo de maquinaria es tener un funcionamiento ininterrumpido, con el pasar del tiempo su actividad decae y es necesario cambiar la fuente radioactiva, resguardarla del medio ambiente, pues esta representa un peligro para la poblacion y un problema es la acumulacion de estos materiales, ya que la contaminación es latente.\cite{C1}\\

Otro de los tantos ejemplos de las aplicaciones de estas tecnologías es en la rama de la fusión nuclear, ya que estas requieren fuentes de alto voltaje para su funcionamiento, en Mexico existe desarrollo en esta área. En la parte experimental, en 1978 se inició un proyecto mexicano de fusión termonuclear y, en 1983, se propuso el diseño de una pequeña máquina experimental llamada “Novillo”\cite{novillo}. Este Tokamak fue diseñado y construido por trabajadores mexicanos del ININ (Instituto Nacional de Investigaciones Nucleares)\cite{inin} en el Centro Nuclear de Salazar, México. Este tipo de maquinas  permitio prticipar en una de las áreas de investigación en Física de Plasmas más prometedoras para el futuro energético. La infraestructura existente y la experiencia adquirida, permitirán contribuir al desarrollo de una futura aplicación de la energía nuclear de fusión, la cual será una fuente alterna de energía en el presente siglo. \\

Bajo la premisa de la ventaja del desarrollo tecnológico de los aceleradores de partículas para nuestro país, es necesario comenzar los estudios en estos temas, ya que las posibilidades de aplicación son bastas y de suma importancia. Con el pasar del tiempo las aplicaciones han aumentado considerablemente, desde ramas de la medicina como ya lo mencionamos, hasta sistemas de aislamiento por campo magnético de plasmas y sistemas para aumentar temperaturas hasta puntos de fusión para sistemas de generación de energía en plantas de fusión nuclear, los mexicanos han apostado por la participación en el desarrollo de estas tecnologías.\\

Este trabajo de tesis pretende dar un pequeño acercamiento a temas relacionados con los ya antes mencionados, mediante el desarrollo de la instrumentación de una fuente de alto voltaje para un acelerador de electrones lineal, el cual se divide en varias etapas de desarrollo. La primera es el sistema de fuentes, nos hemos basado en el sistema Cockroft-Walton (CW) (1932) \cite{Cockcroft} combinado con un inversor de voltaje de alta frecuencia incidente en un embobinado de ignición controlado digitalmente mediante un microcontrolador a lazo cerrado, el cual cuenta con una comunicación PC-HPS (HIGH POWER SUPPLY por sus siglas en ingles), permitiendo al usuario, mediante una retroalimentación, configurar la fuente de voltaje a los parámetros deseados, así como también guardar un registro en las variaciones de corriente y voltaje a la que nuestra fuente es sometida. \cite{C2}\\
\newpage


%%%%%%%%%%%%%%%%%%%%%%%%%%%%%%%%%%%%%%%%%%%%%%%%%%%%%%%%%%%%%%%%%%%%%%%%%
%                   Planteamiento del problema                          %
%%%%%%%%%%%%%%%%%%%%%%%%%%%%%%%%%%%%%%%%%%%%%%%%%%%%%%%%%%%%%%%%%%%%%%%%%

\section{Planteamiento del problema}
Hoy en día el desarrollo de tecnologías que involucran aceleradores de partículas esta cada vez mas presentes en la vida diaria, en México ya existe participación en desarrollo de gran nivel, como lo es el Instituto de Investigaciones Nucleares (ININ), lo cual brinda la posibilidad a los investigadores de involucrarse en este tipo de desarrollo para poder satisfacer las necesidades
que se requieren.\\

El uso de aceleradores de partículas para aplicaciones medicas ha tenido gran auge en los últimos años, ya que las ventajas que tienen sobre las fuentes radioactivas son bastas, este hecho da la oportunidad a las universidades de preparar expertos en estos temas y diseñar maquinaria a medida, que cumpla las exigencias de la región. Aunque ya existen trabajos referentes a fuentes de alto voltaje, generación de electrones mediante telurio y detectores de estos mismos, la curva de aprendizaje necesaria para especializarse en estos temas es grande y dejar un precedente en nuestro país es necesaria y muy útil, es por ello que estas
investigaciones son de gran importancia.
\newpage
%%%%%%%%%%%%%%%%%%%%%%%%%%%%%%%%%%%%%%%%%%%%%%%%%%%%%%%%%%%%%%%%%%%%%%%%%
%                           Hipotesis y objetivos                       %
%%%%%%%%%%%%%%%%%%%%%%%%%%%%%%%%%%%%%%%%%%%%%%%%%%%%%%%%%%%%%%%%%%%%%%%%%
\section{Hipótesis y objetivos}

\subsection{Objetivo General}

Este trabajo tiene por objetivo desarrollar una fuente de alto voltaje de hasta 2 KV y 10 W de potencia.

\subsection{Objetivo Particular}


\begin{itemize}
\item Desarrollar de fuente de bajo voltaje.
\begin{itemize}
\item Diseño de fuente de voltaje regulable a 180W.
\item Simulación de fuente de voltaje regulable.
\item Maquinado de fuente de voltaje regulable.
\end{itemize}
\end{itemize}


\begin{itemize}
\item Inversor de voltaje.
\begin{itemize}  
\item Diseño de driver modulador de ancho de pulso (PWM) bipolar para inversor.
\item Simulación de driver generador de PWM.
\item Construcción de driver (PWM) bipolar.
\item Diseño de interfaces gráficas para control de inversor.
\end{itemize} 
\end{itemize}  

\begin{itemize}
\item Etapa de alto voltaje.
\begin{itemize}
\item Transformadores de alto voltaje.
\item Multiplicador de voltaje.
\end{itemize}
\end{itemize}

\subsection{Hipótesis}
El diseño adecuado de un sistema de generación de alto voltaje va a permitir el desarrollo de aplicaciones en aceleradores lineales. 

\newpage

%%%%%%%%%%%%%%%%%%%%%%%%%%%%%%%%%%%%%%%%%%%%%%%%%%%%%%%%%%%%%%%%%%%%%%%%%
%                           Estructura de la tesis                      %
%%%%%%%%%%%%%%%%%%%%%%%%%%%%%%%%%%%%%%%%%%%%%%%%%%%%%%%%%%%%%%%%%%%%%%%%%

\section{Estructura de la tesis}

Este trabajo está dividido en 5 capítulos. El primero habla sobre un resumen del trabajo realizado, el segundo es un marco teórico que pone en contexto el desarrollo del proyecto y expone las partes técnicas, en el capítulo 3 se expone el proceso y la metodología realizada, el capítulo 4 incluye un breve análisis de resultados el cual muestra una síntesis de las mediciones y el trabajo teórico, y por último un análisis de resultados.             % ~10 páginas - Explicar el propósito de la tesis

%%%%%%%%%%%%%%%%%%%%%%%%%%%%%%%%%%%%%%%%%%%%%%%%%%%%%%%%%%%%%%%%%%%%%%%%%
%           Capítulo 2: MARCO TEÓRICO - REVISIÓN DE LITERATURA
%%%%%%%%%%%%%%%%%%%%%%%%%%%%%%%%%%%%%%%%%%%%%%%%%%%%%%%%%%%%%%%%%%%%%%%%%

\chapter{Marco teórico}

\section{Calculo de capacidad de corriente en pistas de circuitos impresos}



Antes de comenzar con la fabricación de un diseño de PCBs se debe de considerar el tamaño de pistas necesarios para el manejo de corrientes para cada circuito desarrollado, por esa razón mediante un análisis empírico se debe avanzar en el diseño.\\

En la actualidad los requerimientos de corriente llevan al límite la capacidad de reducir el ancho de las pistas y espacios debido a que se desarrollan componentes cada vez más pequeños y sistemas igualmente más compactos, esto obliga al desarrollador a adaptarse a estos nuevos requerimientos.\\

Para encontrar una solución a esta eventualidad es necesario recurrir a estudios en estos temas que nos permita acercarnos al límite y para ello debemos de considerar todos los parámetros que influyan en nuestro sistema, obteniendo así resultados más precisos. En nuestro caso nos basaremos en los gráficos publicados en el IPC2152 ``Standard for Determining Current Carry Capacity in Printed Board Design'' en 2009, este estándar es ampliamente utilizado en muchos proyectos que requieran este tipo de análisis. \\

Para el correcto entendimiento de los procesos que influyen en las pistas por el paso de la corriente debemos de recordar que el paso de la corriente por un conductor produce en este una caída de potencial que esta gobernada por la ley de OHM (R=V/I), esta caída de potencial se disipa en forma de calor por el efecto Joule $Q=I^{2}Rt$. En nuestro el conductor es nuestra pista, su resistencia depende de varios factores, pero lo principal es su sección (ancho x espesor) y su longitud. El efecto térmico es en realidad el que nos interesa conocer al momento del dimensionamiento de la PCB. Por esta razón, para poder calcular una capacidad de transporte de corriente, hay que analizarlo en términos de incremento de temperatura. Fijando como un incremento maxico admisible.\\

Existen algunos parámetros que se deben de considerar importantes de conocer, ya que los mismos alteran o modifican el comportamiento termico de la pista, afectando de manera significativa, los mas importantes son:\\

\begin{itemize}
\item Corriente eléctrica que circula.
\item Tipo de material base.
\item Calculo de corriente de pistas.
\item Sección de la pista.
\item Espesor del laminado de cobre.
\item Espesor de la placa.
\item Presencia de planos de tierra o grandes áreas de cobre.
\item Ambiente de aplicación (gabinete, forzadores de aire, vacío, etc.)
\end{itemize}

Considerar todos estos parámetros en un modelo es bastante complicado, tanto que, en sí, el estándar fue fijado por medio de ensayos y presentando los resultados en forma de curvas. Mediante estos datos empíricos se hace una aproximación que se acerque al límite que deseamos, tomando en cuenta que es importante sobredimensionar dichos límites. \\

El cálculo que se realiza se basa en el fijado de una variación máxima de temperaturas admisibles. La variación térmica se define como un aumento de temperatura por encima de la temperatura inicial que experimenta el conductor. \\
Para el cálculo se requieren los gráficos ya antes mencionados que son dos. El primer grafico es una de las tres entradas y se trata de una serie de curvas que corresponden a los incrementos de temperatura desde diez a cien grados centígrados. En el eje de las ordenadas se grafica la corriente máxima en amperes y en el de las abscisas obtenemos la sección de la pista en milésimas de pulgada cuadrada. El segundo grafico tiene de igual manera tres entradas y en esta se centra en el espesor del cobre, adoptando los valores típicos en los que se fabrican las PCBs, llegando desde 0.5 hasta 3 Oz/ft2.\\

Los cálculos necesarios son sencillos y claros de realizar, para ello necesitaremos los siguientes datos:\\

\begin{itemize} 
\item Corriente máxima a soportar.
\item Incremento máximo de temperatura admisible.
\item Espesor de cobre del material utilizado.
\end{itemize}
Utilizando el valor de corriente nos ubicamos en el grafico 1 por el eje de las ordenadas y proyectamos el valor en forma paralela al eje de las abscisas hasta interceptar la curva que corresponde a la temperatura máxima admisible, luego tomamos el punto en las ordenadas hasta obtener el valor de las absisas que le corresponde. Ese valor es el valor de la sección cuadrada que debe de tener la pista.

\begin{figure}[H]
\centering
\includegraphics[width=12cm]{capitulo2/figs/figura1.png}
\caption{Calculo de ancho de pistas 1}
\end{figure}

\begin{figure}[H]
\centering
\includegraphics[width=12cm]{capitulo2/figs/figura2.png}
\caption{Calculo de ancho de pistas 1}
\end{figure}




\newpage
\section{Ruidos e interferencias}
\subsection{Ruidos en micro-controladores y sistemas digitales}

En particular no existe una forma tal cual para evitar ruidos en nuestros sistemas electrónicos, sino más bien dependiendo de las características de los proyectos se siguen distintos arreglos que mejoran la calidad de las señales que muchas veces, sin el trato adecuado pueden afectar a microcontroladores, PICs o distintos sistemas digitales o analógicos que se presenten. \\

Cuando diseñamos un circuito en general, es probable que funcione correctamente en un simulador o en el Protoboard, pero al momento de accionar o activar cargas de potencia o en señales de alta frecuencia los problemas de ruidos aparecen y las afectaciones pueden llegar a distorsionar una señal o causar estragos en nuestros componentes o cargas.\\

Debemos aclarar que los ruidos electrónicos, no afectan a todos los circuitos por igual, debemos de considerar el tipo de electrónica que estamos desarrollando, entre las que se encuentran circuitos de potencia, de alta frecuencia, componentes como microcontroladores, FPGAs, PICs, etc. que requieren estabilidad en alimentaciones o entradas. Considerando esto la eliminación de ruidos puede ser algo simple como condensadores cerámicos o electrolíticos, hasta la necesidad de realizar arreglos complicados.\\

En nuestro caso estaremos accionando transistores, MOSFETs, bobinas y componentes que requieren grandes consumos de corriente, es por ello que necesitamos tomar en cuenta ciertos aspectos para evitar que dicho ruido afecte al sistema en general o a componentes en específicos claves como el microcontrolador o el sistema de conexión al computador que se estará manejando.\\

Para resolver el problema con los ruidos electrónicos, debemos de actuar de forma pro-activa, es decir, tener en cuenta todos los detalles importantes de operación de nuestro circuito, antes de construirlo de forma definitiva. Para ello tenemos que tomar en cuenta que nuestro micro-controlador es un elemento digital y se deben de tomar todos los cuidados para un óptimo funcionamiento.\\

Teniendo en cuenta lo antes mencionado nos apegaremos a una lista de requerimientos que nuestro circuito deberá de cumplir para obtener el menor ruido posible. Nos concentraremos en esta primera lista en requerimientos necesarios para componentes digitales como micro-controladores.

\begin{itemize} 
\item Utilizar un condensador Bypass (0.1uF) entre los pines de alimentación y tierra del microcontrolador y de cada circuito integrado que componga al sistema.
\begin{figure}[H]
\centering
\includegraphics[width=8cm]{capitulo2/figs/fig3.png}
\caption{Condensador Bypass}
\end{figure}

\item Evitar dejar pines sin conexión. Llevarlos a GND, programando dichos pines como salidas y otorgándoles un valor de 0.

\item Utilizar condensadores de aterrizado del cristal.

\item Utilizar Reset por Hardware, ya que este es más efectivo y estable, que el Reset por software.

\item Si el microcontrolador debe leer botones, pulsadores y/o interruptores. Conecte un condensador de 0.1uF entre el pin de entrada y tierra (GND), para eliminar el efecto “antena”, que producen los pines de entrada, del microcontrolador.
\begin{figure}[H]
\centering
\includegraphics[width=12cm]{capitulo2/figs/fig4.png}
\caption{Condensador Bypass}
\end{figure}

\end{itemize}

Para circuitos mas complejos se deben de llevar acabo ciertas consideraciones dependiendo de los niveles de potencia a manejar. En nuestro caso 


\newpage
\section{Fuente de voltaje lineal}
Es común en proyectos de electrónica no especializados la utilización de diferentes tipos de fuentes de voltaje, entre las que se encuentran fuentes lineales, conmutadas, boost o tipo buck y los problemas que puede causar la falta de atención en este punto tan crucial puede afectan los resultados finales de un proyecto. Es por ello que se necesita conocer los principios fundamentales que reinan a este tipo de sistemas que gobernaran el comportamiento de nuestro proyecto al nivel mas básico.\\

La fuente de voltaje lineal consiste en un sistema sencillo y estructurado, el cual se diseña en diferentes configuraciones en cada modulo a partir del tipo de carga que requiere el proyecto. Para ello podemos observar en la figura 2.5 de manera ilustrativa el orden de la estructura básica de una fuente lineal.

\begin{figure}[H]
 \centering
 \includegraphics[width=12cm]{capitulo2/figs/fuentelineal.jpg}
 \caption{estructura fuente lineal}
 \end{figure}
 
 De manera independiente podemos analizar cada aspecto presentado en la imagen, el cual, de uno en uno se va realizando un análisis para definir los valores y topologias que satisfacen las necesidades requeridas. Tomando en cuenta lo mencionado podemos comenzar a definir las ecuaciones y modelos existentes.
\subsection{Transformador}
\subsection{Rectificador}

\subsection{Filtro}
\subsection{regulador}
\subsection{carga}


\newpage
\section{Disipadores de calor}
\section{Micro-controlador}
sda
\section{Inversores de voltaje}
\section{Bobina de ignición}
das
\section{Fuentes de alto voltaje mas comunes}


\section{Multiplicador de voltaje Cockcroft-Walton}
fgdgf           % ~20 páginas - Poner un contexto a la tesis, hacer referencia a trabajos actuales en el tema

%%%%%%%%%%%%%%%%%%%%%%%%%%%%%%%%%%%%%%%%%%%%%%%%%%%%%%%%%%%%%%%%%%%%%%%%%
%           Capítulo 3: Metodologia                   %
%%%%%%%%%%%%%%%%%%%%%%%%%%%%%%%%%%%%%%%%%%%%%%%%%%%%%%%%%%%%%%%%%%%%%%%%%

\chapter{Metodología}
En este capítulo se expone todo el desarrollo de la fuente de alto voltaje en cuestión; diseño del sistema, fabricación del sistema, diseño del firmware del micro controlador, implementación y por último el método experimental.\\

El sistema esta compuesto por dos partes, hardware y firmware, el hardware se compone de tres partes como vemos en la figura 3.1: parte digital de control, inversor de voltaje y rectificador multiplicador. \\

Por otro lado el firmware consiste en los programas que realizan el control completo de la generación de alto voltaje por medio del desarrollo de un ambiente gráfico al que el usuario tiene acceso,  control para el senseo de voltajes, corrientes y protecciones necesarias para el correcto funcionamiento del sistema en general, así como también un control para el inversor de voltaje, mediante la implementación de un ADC ( Analog to Digital Converter) a la salida de una devanado de baja en el transformador de alto voltaje el cual controla la salida de alto voltaje. Observemos de manera gráfica la topologia del sistema en la figura 3.1. 



 % The \cite command functions as follows:
 %   \citet{key} ==>>                Jones et al. (1990)
 %   \citet*{key} ==>>               Jones, Baker, and Smith (1990)
 %   \citep{key} ==>>                (Jones et al., 1990)
 %   \citep*{key} ==>>               (Jones, Baker, and Smith, 1990)
 %   \citep[chap. 2]{key} ==>>       (Jones et al., 1990, chap. 2)
 %   \citep[e.g.][]{key} ==>>        (e.g. Jones et al., 1990)
 %   \citep[e.g.][p. 32]{key} ==>>   (e.g. Jones et al., p. 32)
 %   \citeauthor{key} ==>>           Jones et al.
 %   \citeauthor*{key} ==>>          Jones, Baker, and Smith
 %   \citeyear{key} ==>>             1990





%%%%%%%%%%%%%%%%%%%%%%%%%%%%%%%%%%%%%%%%%%%%%%%%%%%%%%%%%%%%%%%%%%%%%%%%%
%                          Descripción de la planta                     %
%%%%%%%%%%%%%%%%%%%%%%%%%%%%%%%%%%%%%%%%%%%%%%%%%%%%%%%%%%%%%%%%%%%%%%%%%
\section{Diseño del hardware}


La figura 3.1 muestra un diagrama a bloques de la estructura general del hardware que conforma el sistema de generación de alto voltaje, el cual esta compuesto, desde la parte superior a la inferior, primeramente por bloques relacionados con el control digital del sistema, este bloque se encarga de las interfaces para el usuario así como también de el control e instrumentación de los diferentes sensores, el siguiente conjunto de bloques representan la electrónica encargada de la inversión de voltaje y por ultimo tenemos la rectificación.  \\

\begin{figure}[H]
\centering
\includegraphics[width=12
cm]{Capitulo3/figs/diagrama.png}
\caption{Topologia de fuente de alto voltaje}
\end{figure}
\subsection{Hardware de interface}
Para el desarrollo de la interface gráfica se a utilizado un microcontrolador ATMEGA2560, implementado por la facilidad de programación y los tiempos cortos para la conclusión de este proyecto, así como también la implementación de una pantalla TFT-LCD ( Pantalla de cristal líquido de transistores de película fina) y comunicación UART como interfaces gráfica al usuario. Se ha utilizado el hardware de dicha placa y ahorrado tiempo de desarrollo. \\


\begin{figure}[H]
\centering
\includegraphics[width=9
cm]{Capitulo3/figs/pantalla0.jpg}
\caption{LCD-TFT para interface gráfica}
\end{figure}
\subsection{Hardware de fuente de voltaje a 180w}

Esta sección consiste en varias etapas de desarrollo, para ello primero se ha desarrollado una fuente de voltaje de 180W, que es el primer circuito a analizar. Podemos observar en la figura 3.2 el diseño propuesto. El cual esta conformado por el regulador de voltaje LM723 en modalidad fuente de voltaje por modalidad de regulación positiva.\\

Se ha simulado esta fuente de voltaje en el programa LTSPICE como se muestra en la figura 3.4, en este punto se busca el menor riso posible en nuestra salida final, ya que, en este punto el ruido sera amplificado cientos de veces. Podemos observar en la figura 3.5 que nuestra simulación se acerca bastante a lo buscado, una señal de 12v en corriente directa, teniendo una señal estabilizada en un tiempo de 24ms, suficientemente pequeño para la inicialización de nuestro proyecto. El diseño del PCB se encuentra en la figura 3.6, el cual fue diseñado en el programa EAGLE.\\
\begin{figure}[H]
\centering
\includegraphics[width=10cm]{Capitulo3/figs/fuente.png}
\caption{Topologia de fuente de alto voltaje}
\end{figure}


\begin{figure}[H]
\centering
\includegraphics[width=12cm]{Capitulo3/figs/SIMFUENTE.png}
\caption{Simulación en LTSPICE fuente 180w}
\end{figure}




\begin{figure}[H]
\centering
\includegraphics[width=12cm]{Capitulo3/figs/12v.png}
\caption{Simulación fuente regulable en LTSPICE}
\end{figure}

\begin{figure}[H]
\centering
\includegraphics[width=12cm]{Capitulo3/figs/pcb.png}
\caption{Diseño placa de fuente de regulable de 180w}
\end{figure}

\subsection{Hardware inversor}

Para el inversor se ha implementado una topologia del tipo puente H, ya que es una de las mas utilizadas por los desarrolladores. El integrado BTS7960B es un mosfet de potencia bastante robusto y con una comunidad de desarrollo bastante grande, es por ello que se ha seleccionado este modelo entre los miles que existen en el mercado. Podemos observar el diagrama implementado en la figura 3.7 \cite{ibt} y partiendo de el se ha utilizado la placa de desarrollo IBT2 por cuestión de costos. \\

\begin{figure}[H]
\centering
\includegraphics[width=6cm]{Capitulo3/figs/ibt2.png}
\caption{diagrama puente H}
\end{figure}

\section{Firmware}
\subsection{Interface gráfica}
Para el desarrollo de la interface gráfica se ha realizado en el ambiente de programación de Arduino, intentando la utilización de la menor cantidad de librerías de autoria no propia y siguiendo algunas reglas de programación básicas para micro-controladores como lo es la no utilización de los comando delay. Dicho código se divide en varias secciones, para el cual solo se utilizaron las siguientes librerías:

\begin{verbatim}

#include <UTFT.h>
#include <URTouch.h>

\end{verbatim}

En la siguiente figura podemos observar la topologia del firmware que se ha desarrollado.\\

Todo el código esta dividido en funciones, las cuales llamamos en nuestro LOOP, tratando siempre de cumplir con las siguientes características: no utilización de la función delay, no utilización de ciclos que dependa de alguna condición externa, utilizar el menor código posible para una acción. Las funciones que se utilizaron para la el despliegue de información de la primera pantalla fue el siguiente:


\begin{verbatim}

 
void botones1(){ 
  myGLCD.setFont(BigFont); 
  for (x=0; x<3; x++)
  {
    myGLCD.setColor(0, 0, 255);
    myGLCD.fillRoundRect (200, 10+(x*55), 310, 60+(x*55));
    myGLCD.setColor(255, 255, 255);
    myGLCD.drawRoundRect (200, 10+(x*55), 310, 60+(x*55));
  }
for (x=0; x<2; x++)
  {
    myGLCD.setColor(0, 0, 255);
    myGLCD.fillRoundRect (10+(x*155), 175, 155+(x*155), 225);
    myGLCD.setColor(255, 255, 255);
    myGLCD.drawRoundRect (10+(x*155), 175, 155+(x*155), 225);
  }
  myGLCD.setBackColor(0, 0, 255);
  myGLCD.print("ON 2", 220 , 30);
  myGLCD.print("ON 3", 220 , 85);
  myGLCD.print("ON 1", 220 , 140);
  //myGLCD.print("UART ON", 185 , 195);
  myGLCD.print("V SET", 40 , 190);
  myGLCD.print("CONFIG", 190 , 190);
  }
  
void marco1(int x1, int y1, int x2, int y2){ 
  myGLCD.setColor(255, 0, 0);
  myGLCD.drawRoundRect (x1, y1, x2, y2);
  while (myTouch.dataAvailable())
  myTouch.read();
  myGLCD.setColor(255, 255, 255);
  myGLCD.drawRoundRect (x1, y1, x2, y2);
}
\end{verbatim}

Mediante el código anterior podemos, con ciertas variables, dibujar nuestra área de trabajo en la pantalla, obteniendo como resultado el dibujo de la figura 3.6.

\begin{figure}[H]
\centering
\includegraphics[width=8cm]{Capitulo3/figs/pantalla1.jpg}
\caption{Pantalla 1}
\end{figure}

Mediante esta configuración de dibujo partimos para el código de configuración del TOUCH para lo que llamamos la "pantalla 1".

\begin{verbatim}
void touch1(){ 
      myTouch.read();
      x=myTouch.getX();
      y=myTouch.getY();
      if((x>=200) && (x<=310))
      {
        if((y>=10) && (y<=60)){ //boton ON 2
          marco1(200,10,310,60);
        }
        if((y>=65) && (y<=115)){ //boton ON 3
          marco1(200,65,310,115);
        }
        if((y>=120) && (y<=170)){ //boton ON 1
          marco1(200,120,310,170);
        }
      }
      if((y>=175) && (y<=225))
      {
        if((x>=10) && (x<=155)){ //boton V SET
          marco1(10,175,155,225);
          pantalla =2;
        }
      
        if((x>=165) && (x<=310)){ //boton config
          marco1(165,175,310,225);
        }
}
}
\end{verbatim}

Observamos que el despliegue de estas funciones solo están conformadas por elementos "if" y el llamado a funciones descritas por nosotros se despliegan de la misma manera, resaltando esto debido a que se desarrollo un código lo mas eficientemente posible en cuestión de tiempos de ejecución. \\

Dividimos el dibujo de la "pantalla 2" y las funciones para el touch de la pantalla dos en los siguientes codigos:\\

Funciones dibujo pantalla 2

\begin{verbatim}
void botones2(){
  myGLCD.setBackColor(0,0,255);
  for (x=0; x<4; x++) //botones +
  {
    myGLCD.setColor(0, 0, 255);
    myGLCD.fillRoundRect (10+(x*60), 10, 60+(x*60), 60);
    myGLCD.setColor(255, 255, 255);
    myGLCD.drawRoundRect (10+(x*60), 10, 60+(x*60), 60);
    myGLCD.print("+", 27+(x*60), 27);
  }


  for (x=0; x<4; x++) //botones -
  {
    myGLCD.setColor(0, 0, 255);
    myGLCD.fillRoundRect (10+(x*60), 170, 60+(x*60), 220);
    myGLCD.setColor(255, 255, 255);
    myGLCD.drawRoundRect (10+(x*60), 170, 60+(x*60), 220);
    myGLCD.print("-", 27+(x*60), 190);
  }

  for (x=0; x<4; x++) //blanco 
  {
    myGLCD.setColor(255, 255, 255);
    myGLCD.fillRoundRect (10+(x*60), 70, 60+(x*60), 160);
    myGLCD.setColor(255, 0, 0);
    myGLCD.drawRoundRect (10+(x*60), 70, 60+(x*60), 160);
  //  myGLCD.print(p, 27+(x*60), 170);
  }

  myGLCD.setColor(0, 0, 255); /// boton set
   myGLCD.fillRoundRect(250,70,310,160);
   myGLCD.setColor(255, 255, 255);
   myGLCD.drawRoundRect(250,70,310,160);
   myGLCD.print("set" , 255,105);
  }
\end{verbatim}

Podemos observar el resultado del dibujo en la figura XXX.

\begin{figure}[H]
\centering
\includegraphics[width=8cm]{Capitulo3/figs/pantalla2.jpg}
\caption{Pantalla 2}
\end{figure}

 Funciones TOUCH pantalla 2:
\begin{verbatim}
void touch2(){
    myTouch.read();
      x=myTouch.getX();
      y=myTouch.getY();

      if((y>=10) && (y<=60)){ /////////////botones +
        if((x>=10) && (x<=60)){ //boton + kilos
          marco1(10,10,60,60);
          
          myGLCD.setFont(SevenSegNumFont); 
          suma(1,0,0,0,0,0);
        }

        if((x>=70) && (x<=120)){ //boton + centena
          marco1(70,10,120,60);
          suma(0,1,0,0,0,0);
          
          
        }

        if((x>=130) && (x<=180)){ //boton + decena
          marco1(130,10,180,60); 
          suma(0,0,1,0,0,0);
          
        }

        if((x>=190) && (x<=240)){ //boton + unidad
          marco1(190,10,240,60);
          suma(0,0,0,1,0,0);
          
        }
        }
        if((y>=170) && (y<=220)){ ////////////botones -
          
        if((x>=10) && (x<=60)){ //boton + kilos
          marco1(10,170,60,220);
          suma(1,0,0,0,0,1);
        }

        if((x>=70) && (x<=120)){ //boton + centena
          marco1(70,170,120,220);
          suma(0,1,0,0,0,1);
        }

        if((x>=130) && (x<=180)){ //boton + decena
          marco1(130,170,180,220);
          suma(0,0,1,0,0,1);
        }

        if((x>=190) && (x<=240)){ //boton + unidad
          marco1(190,170,240,220);
          suma(0,0,0,1,0,1);
        }
        }

        if((x>=250) && (x<=310)){ // boton SET

          if((y>=70) && (y<=160)){
            marco1(250,70,310,160);
            pantalla =1;
            if(vout >=0 && vout <=1000){
            Serial.println(vout);
            }
            }    
        }
    }
\end{verbatim}

Una vez que definimos las funciones a utilizar para estas dos primeras pantallas proseguimos a las funciones de cálculos y procesamiento de datos. Para ello hemos creado una funcion capaz de configurar el voltaje de salida, manteniendo una comunicación UART hacia un micro-controlador. 

\begin{verbatim}
void suma(int x1,int x2,int x3,int x4, int k, int w){ 
          myGLCD.setFont(SevenSegNumFont);
          myGLCD.setColor(0, 0, 0);
          myGLCD.setBackColor(255,255,255);
          int q;  
          if(x1 == 1){//algoritmo kilos
            if(w==0 && p<9){
            vout=vout+1000;
            p=p+1;
            q=p*x1+p1*x2+p2*x3+p3*x4;
            sprintf(dato,"%d",q);
            }
            if(w==1 && p>0){
            vout=vout-1000;
            p=p-1;
            q=p*x1+p1*x2+p2*x3+p3*x4;
            sprintf(dato,"%d",q);
              
              }
            myGLCD.print(dato,20*x1+80*x2+140*x3+200*x4,90);
          }
          if(x2 == 1){//algoritmo centena
            if(w==0 && p1<9){
            vout=vout+100;
            p1=p1+1;
            q=p*x1+p1*x2+p2*x3+p3*x4;
            sprintf(dato,"%d",q);
            }
            if(w==1 && p1>0){
            vout=vout-100;
            p1=p1-1;
            q=p*x1+p1*x2+p2*x3+p3*x4;
            sprintf(dato,"%d",q);
              
              }
            myGLCD.print(dato,20*x1+80*x2+140*x3+200*x4,90);
          }
          if(x3 == 1){//algoritmo decenas
            if(w==0 && p2<9){
            vout=vout+10;
            p2=p2+1;
            q=p*x1+p1*x2+p2*x3+p3*x4;
            sprintf(dato,"%d",q);
            }
            if(w==1 && p2>0){
            vout=vout-10;
            p2=p2-1;
            q=p*x1+p1*x2+p2*x3+p3*x4;
            sprintf(dato,"%d",q);
              
              }
            myGLCD.print(dato,20*x1+80*x2+140*x3+200*x4,90);
            //Serial.println(vout);
          }
          if(x4 == 1){//algoritmo unidades
            if(w==0 && p3<9){
            vout=vout+1;
            p3=p3+1;
            q=p*x1+p1*x2+p2*x3+p3*x4;
            sprintf(dato,"%d",q);
            }
            if(w==1 && p3>0){
            vout=vout-1;
            p3=p3-1;
            q=p*x1+p1*x2+p2*x3+p3*x4;
            sprintf(dato,"%d",q);
              
              }
            myGLCD.print(dato,20*x1+80*x2+140*x3+200*x4,90);
          }
          
          if(k==1){
            for(x=0 ; x<4 ; x++){
              sprintf(dato,"%d",p);
              myGLCD.print(dato,20,90);
              sprintf(dato,"%d",p1);
              myGLCD.print(dato,80,90);
              sprintf(dato,"%d",p2);
              myGLCD.print(dato,140,90);
              sprintf(dato,"%d",p3);
              myGLCD.print(dato,200,90);
              }
            }
    }
\end{verbatim}

Una segunda funcion es la encargada de recibir los datos procedentes del segundo micro-controlador, encargado de leer los datos de sensores y preparar el funcionamiento del inversor. 

\begin{verbatim}
void vinput(){
  
      str = Serial.readStringUntil('\n');
      for (int i = 0; i < dataLength ; i++)
      {
         int index = str.indexOf(separator);
         data[i] = str.substring(0, index).toInt();
         str = str.substring(index + 1);
      }
      
      for (int i = 0; i < sizeof(data) / sizeof(data[0]); i++)    
      {
        Serial.print(data[i]); 
      Serial.print('\t');} 
      Serial.println();
      
                  myGLCD.setFont(BigFont);
                  sprintf(vin, "%d",data[0]);
                  sprintf(current, "%d", data[1]);
                  myGLCD.print("     " ,10,50);
                  myGLCD.print("     " ,10,70);
                  myGLCD.setBackColor(0,0,0);
                  myGLCD.setColor(255,255,255);
                  myGLCD.print(vin ,10,50);
                  myGLCD.print(current ,10,70);
  }
\end{verbatim}

Mediante las funciones anteriores podemos mantener una comunicacion INPUT y OUTPUT mediante UART, con el segundo micro-controlador, y mantener un senseo de las variables necesarias para el correcto funcionamiento de la fuente de alto voltaje. Ahora mostramos en cuerpo del programa principal, encargado del control de cada una de las funciones anteriores.

\begin{verbatim}
void loop(){

//pantalla 1
if(pantalla == 1){
    myGLCD.fillScr(VGA_BLACK);
    botones1();
    myGLCD.setFont(BigFont); 
    char set[25];
    sprintf(set, "%d",vout);
    myGLCD.setBackColor(0,0,0);
    myGLCD.print(set ,10,100);
    myGLCD.print("V set" ,100,100);
    myGLCD.print("V out" ,100,50);
    myGLCD.print("mA out" ,100,70);
    myGLCD.print(vin ,10,50);
    myGLCD.print(current ,10,70);
    while(true)
         {
          if(myTouch.dataAvailable())touch1();
          if(pantalla == 2 )break;

      if (Serial.available()>0){vinput();}
            }
            }
          
        
//pantalla 2
if(pantalla == 2){
  myGLCD.fillScr(VGA_BLACK);
  
  botones2();
  
  suma(0,0,0,0,1,0);
  while(true)
    {
      if(myTouch.dataAvailable())touch2();
      if(pantalla == 1)break;
    }
  }


  }
\end{verbatim}      % ~20 páginas - Explicar el problema en específico que se va a resolver, la metodología y experimentos/métodos utilizados
\chapter{Análisis de Resultados}
\section{Resultados}

Se ha logrado mediante la implementación de un sistema de inversor de voltaje alcanzar voltajes de hasta 1 KV a una potencia de 20w, voltaje controlado digitalmente por un computador o de manera manual mediante una interface digital. Se ha desarrollado un código en root CERN para encontrar el voltaje RMS de nuestro voltaje de salida. Nuestras mediciones realizadas arrojaron las  distribuciones de las figuras 3.28, 3.29, 3.30 y 3.31 para diferentes voltajes sin carga alguna. \\.


Como observamos se ha obtenido voltajes sin perturbaciones y con relativo bajo rizo asociado a él en el orden micro. Se realizaron cien mil mediciones por cada distribución y a partir de ella podemos observar un voltaje RMS de 4.3V a 93V, este presenta el mayor rizo, ya que nuestro transformador esta diseñado para manejar altos voltajes, 0.00027V para 200V, 0.0011V para 600V y 0.0024V para 986V respectivamente. \\

Se observa en la tabla los siguientes resultados de nuestras mediciones con sus variables correspondientes. 

\begin{table}[H]
\begin{tabular}{@{}llll@{}}
\toprule
Voltaje   & Frecuencia & Dutty (\%) & Rizo (\%)\\ \midrule
93  & 4.4 KHz        & 4   & $0.043\pm 0.38$\\
200 & 4.05 KHz        & 4   &   $135\mu\pm 0.78$ \\
600 & 4.68 KHz        & 6   &   $185\mu\pm 2.34$ \\
986 & 3.14 KHz       & 7   &   $246.7 \mu \pm 3.85$ \\ \bottomrule
\end{tabular}
\end{table}


Los resultados indican que se ha desarrollado una fuente que ha logrado superar los 1000V de salida, encontrando limitado nuestro trabajo por la resolución de nuestras herramientas de medicion.     % ~20 páginas - Presentar los resultados tal cual son, y analizarlos.
\chapter{Conclusiones}
Se logro la construcción satisfactoria de una fuente de alto voltaje controlada digitalmente mediante un computador y de forma manual mediante un microcontrolador integrado. Los voltajes máximos alcanzados fueron mas de 1KV, limitados por nuestros instrumentos de medición. Con una potencia de hasta 40W y un rizo en el orden de micro volts.\\

La caída de voltaje es el valor que representa el mayor problema en nuestro circuito, siendo necesario el desarrollo de una retroalimentación que ajuste el voltaje en tiempo real, reduciendo así dicha caída. Nuestro inversor y fuente de voltaje de CD fueron diseñados para soportar hasta 180W, así como tambien nuestro transformador de alto voltaje, por lo que el ajuste en potencia de esta retroalimentación se puede realizar digitalmente como futura continuación del proyecto.            % ~5 páginas - Resumir lo que se hizo y lo que no y comentar trabajos futuros sobre el tema

%%%%%%%%%%%%%%%%%%%%%%%%%%%%%%%%%%%%%%%%%%%%%%%%%%%%%
%                   APÉNDICES                       %
%%%%%%%%%%%%%%%%%%%%%%%%%%%%%%%%%%%%%%%%%%%%%%%%%%%%%
\appendix
%% this file is called up by thesis.tex
% content in this file will be fed into the main document
\chapter{Código/Manuales/Publicaciones}
\section{Distribuciones normal}


La importancia de esta distribución radica en que permite modelar numerosos fenómenos naturales. Mientras que los mecanismos que subyacen a gran parte de este tipo de fenómenos son desconocidos, por la enorme cantidad de variables incontrolables  que en ellos intervienen, el uso del modelo normal puede justificarse asumiendo que cada observación se obtiene como la suma de unas pocas causas independientes.\\

De hecho, la estadística descriptiva solo permite describir un fenómeno, sin explicación alguna. Para la explicación causal es preciso el diseño experimental, de ahí que el uso de la estadística como método correlacional.\\

La distribución normal también es importante por su relación con la estimación por mínimos cuadrados, uno de los métodos de estimación mas simples y antiguos.



\section{Señales}
Una señal es continua si no cambia de sentido o polaridad en el periodo de tiempo analizado, aún cuando se haga cero en algún, o algunos, instantes. Caso contrario es clasificado como alterna. Debemos enfatizar que estrictamente esta clasificación es independiente de la ley de variación que tenga; en la jerga técnica suele entenderse como continua a aquella que, además, es constante y como alterna aquella que, además, es senoidal simétrica, pero esto es un hecho particular.

\begin{figure}[H]
\centering
\includegraphics[width=9cm]{continua.png}
\caption{Señal continua y alterna.}
\end{figure}

La segunda clasificación es de constante o variable, siendo constante aquella que no cambia de valor ni sentido en el tiempo y variable en el caso contrario. De hecho una señal constante sólo puede ser continua aunque una continua puede ser constante o variable.\\

Dentro de las variables podemos clasificar a su vez en periódicas o en aleatorias. Periódica es aquella señal en la que puede reconocerse una ley de variación que se repite a intervalos iguales, matemáticamente podemos indicar que $f(t) = f(t+T)$ donde $T$ es el período. Aleatoria es aquella en la que no se encuentra un período de repetición. Esta clasificación es independiente del hecho de ser continua o alterna.

Para dar una idea mejor del tipo de señal a la cual nos estamos refiriendo se indica el nombre que mejor se aproxima a la forma del gráfico representativo. Así es como tenemos ondas sinodales, o armónicas, ondas cuadradas, diente de sierra, etc.

\begin{figure}[H]
\centering
\includegraphics[width=9cm]{periodo.png}
\caption{Señal periódica y aleatorias.}
\end{figure}

\section{Apéndice}

Apéndice
               % Colocar los circuitos, manuales, código fuente, pruebas de teoremas, etc.

%%%%%%%%%%%%%%%%%%%%%%%%%%%%%%%%%%%%%%%%%%%%%%%%%%%%%
%                   REFERENCIAS                     %
%%%%%%%%%%%%%%%%%%%%%%%%%%%%%%%%%%%%%%%%%%%%%%%%%%%%%
% existen varios estilos de bilbiografía, pueden cambiarlos a placer
%\bibliographystyle{apalike} % otros estilos pueden ser abbrv, acm, alpha, apalike, ieeetr, plain, siam, unsrt

%El formato trae otros estilos, o pueden agregar uno que les guste:
%\bibliographystyle{Latex/Classes/PhDbiblio-case} % title forced lower case
%\bibliographystyle{Latex/Classes/PhDbiblio-bold} % title as in bibtex but bold
%\bibliographystyle{Latex/Classes/PhDbiblio-url} % bold + www link if provided
%\bibliographystyle{Latex/Classes/jmb} % calls style file jmb.bst

%% ------------------------------------------------------------------------
% citas y referencias
% ------------------------------------------------------------------------

\begin{thebibliography}{0}
  \bibitem{hola} Mauricio. Algun trabajo, 2016.
  \bibitem{Luckie2010} Matthew Luckie. CScamper: a scalable, extensible packet 
                              prober for active measurement of the internet, 2010.
\end{thebibliography}% Archivo .bib
\begin{thebibliography}{2}
  \bibitem{ProyectoLNLS5} Línea de luz Sincrotrón en México
Acelerador lineal de electrones. \url{https://www.fis.cinvestav.mx/~sincrotron/downloads/ProyectoLNLS5.pdf}

\bibitem{ININ}Instituto Nacional de Investigaciones Nucleares. FUSIÓN NUCLEAR. \url{http://www.inin.gob.mx/temasdeinteres/fusionnuclear.cfm}

\bibitem{IPC 2152}Standard for Determining Current-Carrying Capacity In Printed Board Design. \url{http://electronica.ugr.es/~amroldan/cursos/2014/pcb/modulos/temas/IPC2152.pdf}

\bibitem{transformador}Trasformadores. Miguel Angel Rodríguez Pozueta
Doctor Ingeniero Industrial \url{http://personales.unican.es/rodrigma/PDFs/Trafos.pdf}

\bibitem{rectificador}Rectificador de onda completa. Prof. Julima Anato \url{http://paginas.fisica.uson.mx/horacio.munguia/aula_virtual/Cursos/Instrumentacion\%20I/Documentos/Circuitos_Rectificadores.pdf}

\bibitem{CERN}Multiplicador de voltaje utilizado en CERN en el año 1964. \url{http://cds.cern.ch/record/43889}

\bibitem{ibt}Esquematico de control BTN7970. \url{https://www.elecrow.com/download/IBT-2\%20Schematic.pdf}
\end{thebibliography}


\end{document}
